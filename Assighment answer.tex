\documentclass[12pt, letterpaper]{article}
\usepackage{graphicx}
\usepackage[T1]{fontenc}
\usepackage{float}
\usepackage[table,RGB]{xcolor}
%\usepackage[algo2e,ruled,vlined]{algorithm2e}
%{\color{black}\title{Answer to assignments}} 
\title{Summarized answers to assignment by Latex}
\author{Anders}
\date{February 2023}

\begin{document}
\setlength{\parindent}{0pt}
		
\maketitle
\begin{abstract}
  Total	cities number is 5 for small instances and the cities are marked with 0, 1, 2, 3 and 4. The distances matrix is given at exercise 1(a). The Nearest-Neighbor Heuristic,Extended Nearest-Neighbor Heuristic and Greedy Heuristic were successfully used to solve the ATSP by both manual method and  C++ programming. The C++ coding har been proved very reliable since it gave same result with those which was gotten from manual method. Further, C++ coding were modified to solve large instances and cities number 100 was successfully tested with all three Heuristics.
\end{abstract}
\section{Exercise 1 by manual method}
(a)Create an ATSP instance with n = 5\\

{
	\centering
	
	\begin{tabular}{|l|| c| c| c| c| r|}
    \hline
 	cities and distances & 0 & 1 & 2 & 3 & 4 \\
 	\hline \hline
 	0 & 0 & 11 & 3 & 9 & 15\\
 	\hline
 	1 & 20 & 0 & 12 & 23 &30\\
 	\hline
 	2 &22& 8 &0 &10& 16\\
 	\hline
 	3 &5 &18 &2 &0 &24\\
 	\hline
 	4 &13 &7 &29& 25& 0\\
 	\hline
 	
 	\end{tabular}\par
 
}
\hspace*{\fill}

(b)Apply the Nearest-Neighbor Heuristic with starting city 0:

\begin{center} 
		
	\begin{tabular}{|l|| c| c| c| c| r|}
		\hline
		cities and distances & 0 & 1 & 2 & 3 & 4 \\
		\hline \hline
		0 & 0 & 11 & {\color{blue}3}  & 9 & 15\\
		\hline
		1 & 20 & 0 & 12 & 23 &30\\
		\hline
		2 &22& 8 &0 &10& 16\\
		\hline
		3 &5 &18 &2 &0 &24\\
		\hline
		4 &13 &7 &29& 25& 0\\
		\hline
	\end{tabular}
\end{center}

\begin{center} 
	
	\begin{tabular}{|l|| c|c| c| r|}
		\hline
		cities and distances & (0,2) & 1 & 3 & 4 \\
		\hline \hline
    	(0,2) & 0 & {\color{blue}8}  & 10  & 16\\
		\hline
		1 &20& 0 &23& 30\\
		\hline
		3 &5 &18 &0 &24\\
		\hline
		4 &13 &7 & 25& 0\\
		\hline
	\end{tabular}
\end{center}

\begin{center} 
	
	\begin{tabular}{|l|| c| c| c| r|}
		\hline
		cities and distances & (0,2,1) & 3 & 4 \\
		\hline \hline
		(0,2,1) & 0 & {\color{blue}10}   & 30\\
		\hline
		3 &5 &0 &24\\
		\hline
		4 &13& 25& 0\\
		\hline
	\end{tabular}
\end{center}

\begin{center} 
	
	\begin{tabular}{|l|| c| r|}
		\hline
		cities and distances & (0,2,1,3)  & 4 \\
		\hline \hline
		(0,2,1,3) & 0 & {\color{blue}24} \\
		\hline
		4 &13& 0\\
		\hline
	\end{tabular}
\end{center}
The answer:
the shortest path will be (0, 2 ,1 ,3 ,4, 0)\\

(c)Apply the Extended Nearest-Neighbor Heuristic with starting
city 0


\begin{center} 
	
	\begin{tabular}{|l|| c| c| c| c| r|}
		\hline
		cities and distances & 0 & 1 & 2 & 3 & 4 \\
		\hline \hline
		0 & 0 & 11 & {\color{blue}3}  & 9 & 15\\
		\hline
		1 & 20 & 0 & 12 & 23 &30\\
		\hline
		2 &22& 8 &0 &10& 16\\
		\hline
		3 &5 &18 &2 &0 &24\\
		\hline
		4 &13 &7 &29& 25& 0\\
		\hline
	\end{tabular}
\end{center}


\begin{center} 
	
	\begin{tabular}{|l|| c|c| c| r|}
		\hline
		cities and distances & (0,2) & 1& 3 & 4 \\
		\hline \hline
		(0,2) & 0 & 8  & 10  & 16\\
		\hline
		1 &20& 0 &23& 30\\
		\hline
		3 &{\color{blue}5} &18 &0 &25\\
		\hline
		4 &13 &7 & 25& 0\\
		\hline
	\end{tabular}
\end{center}

\begin{center} 
	
	\begin{tabular}{|l|| c| c| c| r|}
		\hline
		cities and distances & (3,0,2) & 1 & 4 \\
		\hline \hline
		(3,0,2) & 0 & {\color{blue}8} & 16\\
		\hline
		1 &23 &0 &30\\
		\hline
		4 &25& 7& 0\\
		\hline
	\end{tabular}
\end{center}

\begin{center} 
	
	\begin{tabular}{|l|| c| r|}
		\hline
		cities and distances & (3,0,2,1)  & 4 \\
		\hline \hline
		(3,0,2,1) & 0 & 30\\
		\hline
		4 &{\color{blue}25}& 0\\
		\hline
	\end{tabular}
\end{center}
The answer:
the shortest path will be (4,3,0,2,1,4)\\

(d) Apply the Greedy Heuristic with starting vertex 1


\begin{center} 
	
	\begin{tabular}{|l|| c| c| c| c| r|}
		\hline
		cities and distances & 0 & 1 & 2 & 3 & 4 \\
		\hline \hline
		0 & 0 & 11 & 3  & 9 & 15\\
		\hline
		1 & 20 & 0 & 12 & 23 &30\\
		\hline
		2 &22& 8 &0 &10& 16\\
		\hline
		3 &5 &18 &{\color{blue}2} &0 &24\\
		\hline
		4 &13 &7 &29& 25& 0\\
		\hline
	\end{tabular}
\end{center}


\begin{center} 
	
	\begin{tabular}{|l|| c|c| c| r|}
		\hline
		cities and distances & (3,2) & 0 & 1 & 4 \\
		\hline \hline
		(3,2) & 0 & 22  & 10  & 16\\
		\hline
		0 &9& 0 &11& 15\\
		\hline
		1 &23 &20 &0 &30\\
		\hline
		4 &25 &13 & {\color{blue}7}& 0\\
		\hline
	\end{tabular}
\end{center}

\begin{center} 
	
	\begin{tabular}{|l|| c| c| c| r|}
		\hline
		cities and distances & (3,2) & 0 & (4,1) \\
		\hline \hline
     (3,2) & 0 & 22 & 16\\
		\hline
		0 &{\color{blue}9} &0 &15\\
		\hline
		(4,1) &23& 20& 0\\
		\hline
	\end{tabular}
\end{center}

\begin{center} 
	
	\begin{tabular}{|l|| c| r|}
		\hline
		cities and distances & (0,3,2)  & (4,1) \\
		\hline \hline
		(0,3,2) & 0 & {\color{blue}16}\\
		\hline
		(4,1) &20& 0\\
		\hline
	\end{tabular}
\end{center}
The answer:
the shortest path will be (0,3 ,2 ,4 ,1, 0)\\


~\\
\section{Exercise 2 by C++ coding}

  Applying the Nearest-Neighbor Heuristic with starting city 0 related to 1(b):

\begin{figure}[H]
	\centering
	\includegraphics[width=0.7\linewidth]{1}
	\caption{}
	\label{fig:1}
\end{figure}

Applying the Extended Nearest-Neighbor Heuristic with starting
city 0 related to 1(c):
\begin{figure}[H]
	\centering
	\includegraphics[width=0.7\linewidth]{2}
	\caption{}
	\label{fig:2}
\end{figure}
~\\
Applying the Greedy Heuristic with starting vertex 1
related to 1(d):
\begin{figure}[H]
	\centering
	\includegraphics[width=0.7\linewidth]{3}
	\caption{}
	\label{fig:3}
\end{figure}

~\\
\section{Modification for larger instances}

In order to test larger instances, coding were embedded with choices to generate distances matrix by system. If users choose 2, they can get distances matrix 100*100 easily.\\

I have successfully tested with 100 cities and the results were saved to three documents below.\\
Nearest Neighbor Strategy n=100.txt\\
Extended Nearest Neighbor Strategy n=100.txt\\
Greedy Strategy n=100.txt\\


\end{document}
